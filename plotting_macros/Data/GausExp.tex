\documentclass[14pt]{extarticle}
\usepackage{amsmath}
\usepackage{mathtools}
\usepackage{amssymb}
\usepackage{xspace}
\usepackage[margin=1in]{geometry}

\usepackage{graphicx} % \scalebox
\usepackage{environ}
\NewEnviron{myequation}{%
\begin{equation}
\scalebox{1.5}{$\BODY$}
\end{equation}
}
\NewEnviron{myalign}{%
\begin{align}
\scalebox{1.5}{$\BODY$}
\end{align}
}

\renewcommand{\d}{\ensuremath{\mathrm{d}}\xspace}
\newcommand{\ddx}[1]{\ensuremath{\frac{\d #1}{\d x}}\xspace}

\begin{document}

Let \(f(x)\) be the following piece-wise function composed of a Gaussian and an exponential function

\begin{equation}
  f(x) =
  \begin{cases}
    A e^{-\frac{(x - \mu)^2}{2\sigma^2}}, & x < \mu + \lambda \sigma \\
    A e^{-\frac{x - \nu}{\tau}}, & x \geq \mu + \lambda \sigma
  \end{cases}
\end{equation}

where \(A\) is the amplitude, \(\mu\) is the mean, \(\sigma\) is the standard deviation, and \(\lambda\) is the number of standard deviations from the mean where the transition occurs.
Demanding that \(f(x)\) is (singly) differentiable at every point, find expressions for \(\nu\) and \(\tau\).

From differentiability

\begin{align}
  \lim_{x \uparrow \mu + \lambda \sigma} f(x) &= \lim_{x \downarrow \mu + \lambda \sigma} f(x), \\
  \lim_{x \uparrow \mu + \lambda \sigma} \ddx{f}(x) &= \lim_{x \downarrow \mu + \lambda \sigma} \ddx{f}(x),
\end{align}

and so

\begin{align}
  A e^{-\lambda^2/2}
    &= A e^{- \frac{\mu + \lambda \sigma - \nu}{\tau}}, \\
  - \frac{\lambda}{\sigma} A e^{-\lambda^2/2}
    &= - \frac{1}{\tau} A e^{-\frac{\mu + \lambda \sigma - \nu}{\tau}}. \\
\end{align}

Then

\begin{equation}
  - \frac{\lambda}{\sigma} A e^{-\frac{\mu + \lambda \sigma - \nu}{\tau}} = - \frac{1}{\tau} A e^{-\frac{\mu + \lambda \sigma - \nu}{\tau}}
  \Rightarrow \tau = \frac{\sigma}{\lambda}.
\end{equation}

And

\begin{equation}
  e^{\nu/\tau} = \exp\left[\frac{\mu + \lambda \sigma}{\tau} - \frac{\lambda^2}{2}\right]
  \Rightarrow \nu = \mu + \lambda \sigma / 2.
\end{equation}

If, instead \(f(x \geq \mu + \lambda \sigma) = \exp(-(x-\nu)/\tau)\), then \(\nu = \ln(A) \sigma/\lambda + \mu + \lambda \sigma / 2\) and \(\tau = \sigma/\lambda\).

\end{document}